\chapter[Fundamentação Teórica]{Fundamentação Teórica}

\section{Biomassa}

\subsection{Caracterização da Biomassa}

\section{Gaseificação}

Gaseificação é um processo termoquímico que converte um combustível sólido ou líquido em um gás através de oxidação parcial a temperaturas elevadas. A tecnologia consiste em suprir a reação com quantidades restringidas de oxidante, que pode ser oxigênio puro, ar atmosférico ou vapor d’água, produzindo gás de síntese (ou syngas), que é composto, basicamente, por hidrogênio e monóxido de carbono \cite{biomassacortez}.

O gás de síntese, resultado da reação de gaseificação, pode ser aplicado na produção de combustíveis líquidos e na geração de energia mecânica e elétrica \cite{biomassacortez}. Seu poder calorífico é baseado, dentre outros fatores, no agente oxidante utilizado, como pode ser visto na Tabela 1.

\begin{table}[h]
	\centering
	\caption{Poder calorífico do syngas baseado no agente oxidante \cite{basu2010}}
	\begin{tabular}{c|c}
		\textbf{Agente Oxidante} & \textbf{Poder Calorífico (MJ/Nm\textsuperscript{3})} \\
		\hline
		Ar & 4 - 7 \\
		Vapor d'água & 10 - 18 \\
		Oxigênio puro & 12 - 28 
	\end{tabular}
\end{table}	
		
		

Esses valores têm significativa influência nas aplicações do syngas. O vapor e o oxigênio, que têm um poder calorífico médio, são melhor utilizados na produção de combustíveis, já para a geração de energia elétrica através de motores e turbinas, pode-se utilizar um oxidante com baixo poder calorífico, como o ar \cite{bridgwater2003}.

\subsection{Gaseificador}
Os gaseificadores são os equipamentos onde ocorre a gaseificação. Eles são classificados pela direção do movimento da biomassa e do agente oxidante e podem ser divididos em três grupos: leito fixo, que são subdivididos em contracorrente (updraft), concorrente (downdraft) e fluxo cruzado (crossdraft), leito fluidizado e leito arrastado \cite{higman2007}.
 
Os gaseificadores de leito fixo são os de mais simples construção, de maior empregabilidade e são aplicados em sistemas de pequena escala, enquanto os de leito fluidizado e leito arrastado são para média e alta escala, respectivamente \cite{basu2010}. 


\subsubsection{Leito fixo Contracorrente}

Nos gaseificadores contracorrente, a biomassa é introduzida em uma grelha onde desce pela ação da gravidade enquanto o agente oxidante segue o fluxo contrário (downdraft). Possuem eficiência térmica elevada, uma vez que o agente oxidante vem por baixo passando primeiro pela zona de combustão, onde está com mais elevada temperatura, pré-aquecendo a caga de combustível em seu fluxo ascendente. Porém, o syngas apresenta alto teor de alcatrão, que não passa por craqueamento na zona de combustão, o que pode causar incrustações e danos a equipamentos  mecânicos, como o motor, e a tubulações, inviabilizando sua aplicação para produção de energia mecânica e elétrica sem um eficiente processo de limpeza do gás \cite{sanchez2010}. São, portanto, mais adequados à aplicação para fornecimento de calor, como fornos e caldeiras. Também são ideais para biomassas com elevada umidade e concentração de cinzas \cite{basu2010}.



\subsubsection{Leito fixo Concorrente}

Neste gaseificador, o agente oxidante segue o mesmo fluxo da biomassa e é introduzido uniformemente na zona de combustão, região de alta temperatura onde o alcatrão passa por craqueamento, resultando em um gás de síntese com significativo menor teor de alcatrão, tornando-o mais adequado para aplicações na geração de energia mecânica e elétrica. Porém, ao seguir para a zona de redução, o gás adquire maiores quantidades de cinza e fuligem, condicionando-o a também passar por limpeza antes de seguir para o equipamento mecânico \cite{sanchez2010}.


\subsubsection{Leito Fixo de Fluxo Cruzado}

Nestes gaseificadores, o ar é injetado diretamente no centro da zona de combustão, onde, no mesmo patamar, o syngas é retirado a uma velocidade muito rápida e temperaturas extremamente elevadas. São adequados à aplicação na geração de energia mecânica e elétrica por possuir rápida resposta à variação de carga, porém, não se adequam muito ao uso da biomassa por possuir alta sensibilidade à umidade \cite{sanchez2010}.


\subsubsection{Leito Fluidizado e Leito Arrastado}

Nos gaseificadores de leito fluidizado, a biomassa é introduzida em toda a extensão do gaseificador com um material inerte suportados por uma placa distribuidora, o agente oxidante sobe pelo gaseificador de forma ascendente reagindo com a mistura \cite{sanchez2010}e com velocidade suficiente para fazê-la levitar e se espalhar pela grelha até atingir o topo onde a velocidade reduz e as partículas recirculam pela grelha causando alta transferência de massa e calor entre as partículas de combustível e o agente oxidante \cite{reed1988}. 

Nos gaseificadores de leito arrastado, a biomassa e o agente oxidante seguem o mesmo fluxo dentro do gaseificador em poucos segundos. A biomassa é introduzida com um diâmetro igual ou inferior a 100 $\mu$m para que possa ser transportada pelo gás \cite{higman2007}.


\subsection{Processo de gaseificação}

O processo de gaseificação de uma biomassa para geração de energia elétrica pode ser visto na Figura 2.

*************************

O teor de umidade da biomassa é um inconveniente para o processo de gaseificação, pois necessita-se de um mínimo de 2260 kJ para vaporizar 1 kg de água presente na biomassa \cite{basu2010}. Além disso, para o bagaço de cana, quanto maior for o teor dessa umidade, menor é o seu poder calorífico, levando a um menor aproveitamento energético \cite{silva2008}. Dessa forma, é de suma importância que a amostra passe por um processo de secagem antes e ao ser introduzida no gaseificador.

Em um gaseificador downdraft, após a zona do processo de secagem, a biomassa segue para a zona de pirólise, onde 


A qualidade do syngas produzido é determinada pelo processo de gaseificação adotado e pelas características físicas e químicas da biomassa, que, por sua vez, são determinadas pelas análises imediata e elementar \cite{chaves2016}.

Antes de ser encaminhado para o motor, o syngas deve ser previamente tratado devido à presença de alcatrão e cinzas. Para que o motor opere em condições satisfatórias, a concentração de alcatrão no MCI deve ser menor que 100mg/Nm\textsuperscript{3} \cite{hasler1999}.

As tecnologias utilizadas para geração de energia elétrica a partir do syngas produzido no gaseificador podem ser motor de combustão interna (MCI), motor stirling, microturbinas e células a combustíveis, sendo que o motor de combustão interna apresenta maior maturidade tecnológica e comercial \cite{lora2006} e, comparado aos demais, menor custo por kw \cite{chaves2016}.

Estudos e experimentos mostraram sistemas de gaseificação para geração de energia elétrica em pequena escala com resultados satisfatórios e viáveis. \cite{chaves2016} utilizou um sistema com gaseificador downdraft e motor com ignição por faísca acoplado a um gerador alimentado por lascas de madeira, que apresentou uma eficiência no processo de gaseificação de 60-70\%, decaindo para 4,5-17\% de eficiência na geração elétrica, devido ao poder calorífico da biomassa e à eficiência térmica do motor.

\cite{villarini2015} realizou um estudo de caso  em uma fazenda que foi alimentada energeticamente com um sistema composto por gaseificador de leito fluidizado e motor de combustão interna, os resultados mostraram que alimentando 60kg/h de biomassa, obteve-se 56.6 Nm\textsuperscript{3}/h de vazão do syngas produzindo 50 kW de energia elétrica e 91kW de energia térmica, que foram capazes de suprir a energia demandada e vender o excedente.

\cite{figueiredo2012} gaseificou lenha de eucalipto em um sistema de gaseificador downdraft e motor de combustão interna acoplado a um gerador de 50kVA com o objetivo de avaliar a viabilidade de se aplicar o sistema em localidades distantes. A produção de syngas conseguiu suprir a demanda máxima do gerador, consumindo aproximadamente 49,6 kg/h de biomassa. 

\section{Geração Distribuída}