\begin{resumo}[Abstract]
 \begin{otherlanguage*}{english}
Given the current hydric crisis in Brazil and the high percentage of hydroelectric power plants participation in the Brazilian electrical energy supply, which was 61,5\% in 2006, as well as the increasing increase on the country’s energetic demand, a decrease on the dependency of hydric energy becomes necessary. As a solution, there can be incentives on the usage increase of renewable energy alternative sources, like biomass, and incentives to decentralized generation. This way, it is important to execute studies on energetic systems capable of supplying electrical energy to the end user in an efficient, sustainable and technical-economically viable way from clean and renewable sources. This work’s goal is to analyze the feasibility of a sugarcane bagasse gasification system for decentralized electrical energy generation as exploitation of the generated residues from business establishments that commercialize sugarcane broth. This work presents the background information to understand such system operation and the materials and methods that will be used for the experimental studies and the computational simulation.

   \vspace{\onelineskip}
 
   \noindent 
   \textbf{Key-words}: Gasification. Sugarcane bagasse. Electrical Energy Generation.
 \end{otherlanguage*}
\end{resumo}
