\begin{resumo}[Abstract]
 \begin{otherlanguage*}{english}
Given the current hydric crisis in Brazil, the high percentage of hydroelectric power plants participation in the Brazilian electrical energy supply and the increasing increase on the country’s energetic demand, a decrease on the dependency of hydric energy becomes necessary. As a solution, there can be incentives on the usage increase of renewable energy alternative sources, like biomass, and incentives to decentralized generation. The sugarcane bagasse is a biomass widely used as an energetic source in Brazilian territory and commercial establishments that sells sugarcane juice can save financially with the energetic exploitation of this residue that comes from its own production process. In order to analyze the possibility of this savings, this work's goal is to analyze the feasibility of a gasification system, where an internal combustion engine is powered with the gas generated from the sugarcane bagasse's gasification. The experimental study analyzed the reduction on the engine's fuel consumption  and, in a theoretical way, the electric energy generation. The engine showed a fuel consumption reduction of 14,4\% by inserting the synthesis gas and, considering the syngas thermal power with no losses, it is capable of generating 4,39kW operating at dual-fuel mode, which makes its application technically viable. On the other hand, the system is not ecnomically viable to a decentralized energy generation due to the gasoline's high price.

   \vspace{\onelineskip}
 
   \noindent 
   \textbf{Key-words}: Gasification. Sugarcane bagasse. Electrical Energy Generation.
 \end{otherlanguage*}
\end{resumo}
