\begin{resumo}

Tendo em vista as recorrentes crises hídricas no Brasil, o alto percentual da participação das usinas hidrelétricas na oferta de energia elétrica brasileira e o crescente aumento da demanda energética no país, faz-se necessária uma diminuição na dependência que o país tem na energia hídrica. Como solução, pode-se investir no incentivo ao aumento do uso de fontes alternativas de energia renovável, como a biomassa, e o incentivo à geração distribuída. O bagaço de cana de açúcar é uma biomassa amplamente utilizada como fonte energética no território brasileiro e estabelecimentos comerciais, como comércios de caldo de cana, podem obter economia financeira com o aproveitamento energético desse resíduo advindo do seu processo produtivo. A fim de analisar a possibilidade desta economia, este trabalho tem como objetivo analisar a viabilidade de um sistema de gaseificação, onde um motor de combustão interna é alimentado por um gás proveniente da gaseificação do bagaço de cana. O estudo experimental analisou a redução no consumo de combustível do motor e, de forma teórica, a geração de energia elétrica. O motor apresentou uma redução de 14,4\% no consumo de gasolina com a inserção do gás de síntese e é capaz de gerar 4,39kW de potência operando a duplo combustível, o que torna sua aplicação tecnicamente viável. O sistema, porém, não é economicamente viável devido ao alto preço da gasolina. 

 \vspace{\onelineskip}
    
 \noindent
 \textbf{Palavras-chaves}: Gaseificação. Bagaço de cana. Geração de energia elétrica.
\end{resumo}
