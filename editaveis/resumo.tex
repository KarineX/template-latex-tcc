\begin{resumo}

Tendo em vista a atual crise hídrica no Brasil e o alto percentual da participação das usinas hidrelétricas na oferta de energia elétrica brasileira, que em 2006 estava em 61,5\%, bem como o crescente aumento da demanda energética no país, faz-se necessária uma diminuição na dependência da energia hídrica. Como solução, pode-se investir no incentivo ao aumento do uso de fontes alternativas de energia renovável, como a biomassa, e o incentivo à geração distribuída. Dessa forma, é importante que sejam realizados estudos de sistemas energéticos capazes de fornecer energia elétrica ao consumidor de forma eficiente, sustentável e técnica e economicamente viável a partir de fontes renováveis e limpas. O objetivo deste trabalho é analisar o desempenho de um sistema de gaseificação a bagaço de cana para geração distribuída de energia elétrica como aproveitamento do resíduo gerado em estabelecimentos que comercializem caldo de cana. O trabalho apresenta uma fundamentação teórica para entender o funcionamento de tal sistema e os materiais e métodos que serão utilizados para os estudos experimentais e as simulações computacionais.

 \vspace{\onelineskip}
    
 \noindent
 \textbf{Palavras-chaves}: Gaseificação. Bagaço de cana. Geração de energia elétrica.
\end{resumo}
