\begin{resumo}

Tendo em vista a atual crise hídrica no Brasil e o alto percentual da participação das usinas hidrelétricas na oferta de energia elétrica brasileira, que em 2006 estava em 61,5\%, bem como o crescente aumento da demanda energética no país, faz-se necessária uma diminuição na dependência que se tem na energia hídrica. Como solução, pode-se investir no incentivo ao aumento do uso de fontes alternativas de energia renovável, como a biomassa, e o incentivo à geração distribuída. Estabelecimentos comerciais podem obter economia financeira com o aproveitamento energético dos resíduos dos seus processos produtivos como no caso do bagaço de cana de açúcar em comércios de caldo de cana. A fim de analisar a possibilidade desta economia, este trabalho tem como objetivo analisar a viabilidade de um sistema de gaseificação, onde um motor de combustão interna é alimentado por um gás proveniente da gaseificação do bagaço de cana.

 \vspace{\onelineskip}
    
 \noindent
 \textbf{Palavras-chaves}: Gaseificação. Bagaço de cana. Geração de energia elétrica.
\end{resumo}
