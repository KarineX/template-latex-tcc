\chapter[Introdução]{Introdução}

A crescente demanda energética e a preocupação com os impactos ambientais advindos do uso de combustíveis fósseis como fonte de energia têm levado ao incentivo da inserção de fontes alternativas que sejam limpas e renováveis na matriz energética nacional de países em todo o mundo. O Brasil se destaca no cenário mundial por ter uma oferta interna de energia elétrica com ampla participação de fontes renováveis, as quais foram responsáveis por 81,7\% da energia elétrica ofertada no ano de 2016 \cite{mme2017}. Na Figura \ref{figura_matriz}, é possível visualizar o gráfico da Oferta Interna de Energia Elétrica por fonte para o ano de 2016, segundo dados do Ministério de Minas e Energia (MME).

\begin{figure}[!htb]
	\centering
	\includegraphics{OIEE}
	\caption{Oferta Interna de Energia Elétrica por fonte \cite{mme2017}}
	\label{figura_matriz}
\end{figure}

Do total da oferta, 61,5\% são provenientes da energia hídrica, o que torna a matriz energética nacional predominantemente limpa. O percentual das demais fontes renováveis somam 16,3\%, ultrapassando o somatório das não renováveis que totalizam 15,6\%. A energia solar, apesar de ter crescido em relação ao ano anterior, ainda apresenta baixa representatividade, a energia eólica, por outro lado, além do crescimento, apresenta significativa participação na matriz.

Dentre as biomassas, o bagaço da cana de açúcar é o principal responsável pela oferta de energia elétrica nacional, com um percentual de 5,7\%. Sua principal aplicação acontece nas usinas sucroalcooleiras, que o usam em processos de cogeração de energia para a autossuficiência em energias térmica e elétrica, vendendo o excedente deste último para a rede do sistema elétrico nacional. Para esse processo de cogeração, o bagaço resultante da moagem da cana para a obtenção de açúcar e álcool é queimado em uma caldeira e o calor resultante irá aquecer um fluido, geralmente água, gerando vapor que movimentará uma turbina acoplada a um gerador de energia elétrica. Outra forma de obtenção de energia elétrica através da biomassa se dá pelo processo de gaseificação, o qual transforma o combustível sólido em um gás de síntese, que movimentará um motor ou uma turbina acoplados a um gerador. A gaseificação apresenta vantagens sobre a queima direta da biomassa devido à maior eficiencia de conversão energética \cite{chaves2016}. 

Ao contrário do petróleo e seus derivados, a queima do bagaço de cana não é agressiva ao meio ambiente, uma vez que o CO$_2$ emitido é absorvido pela própria plantação para o processo de fotossíntese da planta, sendo considerado, portanto, uma fonte de carbono-neutro \cite{basu2010}.

\section{Justificativa}

Apesar de possuir uma fonte limpa, renovável e com recurso abundante no território brasileiro, a energia hídrica é extremamente dependente de fatores climáticos e meteorológicos. Em períodos de escassez de chuva, os reservatórios das usinas hidrelétricas acumulam baixas porcentagens do seu volume útil com água, o que afeta a geração de energia elétrica para o país. Portanto, é importante que haja um planejamento energético com a finalidade de reduzir a dependência do sistema elétrico brasileiro nas usinas hidrelétricas, bem como atenuar o efeito do crescimento da demanda energética no Sistema Interligado Nacional (SIN). 

Para isso, pode-se investir numa maior diversificação da matriz energética, aumentando o percentual das demais fontes alternativas limpas e renováveis, podendo ser elas solar, eólica, nuclear ou biomassa e no incentivo à geração distribuída, onde o consumidor gere sua própria energia perto do seu local de consumo, diminuindo a demanda do SIN. 

Um estabelecimento comercial que utilize cana-de-açúcar no seu processo produtivo, como a moagem para venda de caldo de cana, tem como resíduo sólido o bagaço, que pode ser utilizado como insumo para geração de energia elétrica para o próprio estabelecimento.  Além de contribuir para a diversificação da matriz energética nacional com uma fonte renovável, o estabelecimento estará agindo de forma ambientalmente correta ao reaproveitar um resíduo sólido resultante de seu processo produtivo, conforme os princípios e objetivos da Política Nacional de Resíduos Sólidos, instituída pela Lei Nº 12.305, de 2 de agosto de 2010 %\cite{pnrs}. 

\section{Objetivos}
\subsection{Objetivo geral}
Analisar o desempenho de um sistema de gaseificação alimentado por bagaço de cana para geração de energia elétrica como aproveitamento energético do resíduo gerado em estabelecimentos que comercializem caldo de cana.

\subsection{Objetivos específicos}
\begin{itemize}
\item Simular as reações de gaseificação do bagaço de cana;

\item Preparar a amostra de biomassa para estudo experimental;

\item Acoplar o reator a um motor para a realização das análises;

\item Analisar o desempenho do sistema;

\item Avaliar se o sistema é viável para aplicação em um estabelecimento comercial.
\end{itemize}



