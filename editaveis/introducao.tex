\chapter[Introdução]{Introdução}


A crescente demanda energética e a preocupação com os impactos ambientais advindos do uso de combustíveis fósseis como fonte de energia têm levado ao incentivo da inserção de fontes alternativas que sejam limpas e renováveis na matriz energética nacional de países em todo o mundo. 

O Brasil se destaca no cenário mundial por ter uma oferta interna de energia elétrica com ampla participação de fontes renováveis, as quais foram responsáveis por 75,5\% da energia elétrica ofertada no ano de 2015 (MME, 2016), sendo que 61,4\% são provenientes da energia hídrica (ANEEL, 2017), o que torna a matriz energética nacional predominantemente limpa.

A energia hídrica, no entanto, é extremamente dependente de fatores climáticos e meteorológicos. Em períodos de escassez de chuva, os reservatórios das usinas hidrelétricas acumulam baixas porcentagens do seu volume útil com água, o que afeta a geração de energia elétrica para o país. Nesses períodos de estiagem, para que a demanda energética seja devidamente atendida, as usinas termelétricas são acionadas.
 
Um quarto das usinas termelétricas brasileiras utilizam a biomassa como combustível, sendo que o bagaço da cana de açúcar é o mais utilizado dentre eles. 

A queima da biomassa não é agressiva ao meio ambiente, uma vez que o CO2 emitido na sua queima é absorvido pela própria plantação para o processo de fotossíntese da planta, caracterizando essa fonte energética como limpa e renovável

A energia solar, apesar de ainda apresentar uma porcentagem muito baixa e pouco significativa na matriz elétrica brasileira, apresentou uma taxa de crescimento de 226,4\% entre os anos de 2014 e 2015 (MME, 2016).

\section{Justificativa}

\section{Objetivos}
\subsection{Objetivo geral}

\subsection{Objetivos específicos}

\section{Estrutura do documento}

