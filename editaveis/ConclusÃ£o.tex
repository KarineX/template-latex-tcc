\chapter[Conclusão]{Conclusão}

O gás de síntese gerado pela gaseificação do bagaço de cana neste experimento acarretou em baixas porcentages de redução no consumo de gasolina do motor utilizado, e uma redução quase nula quando o reator trabalhava a topo aberto.

Para que o motor possa apresentar redução mais significativa no consumo de combustível, é necessário que seja montado um aparato experimental com menos perda de carga. Para isso, ao invés de utiizar o vácuo do motor para causar a depressão na saída do reator, pode-se construir um tubo venturi na entrada de admissão de ar do motor com um orifício maior, o que poderá reduzir a perda de carga. Com o Venturi, a velocidade do ar aumentará na região da garganta do mesmo, causando a depressão necessária e succionando o gás de síntese.

O sistema de gaseificação montado é viável tecnicamente, pois tem condições de fornecer potência suficiente para moendas de diferentes modelos com o motor trabalhando a duplo combustível, sendo o motor acoplado tanto diretamente ao eixo da moenda como acoplado a um gerador elétrico. Porém, este sistema não é viavelmente economicamente com um alto preço da gasolina, pois este aumenta o custo mensal de consumo de energia.


Em uma análise teórica e desprezadas as perdas de acoplamento do sistema de limpeza ao motor, a potência térmica do gás de síntese gerado apresentou valor satisfatório para geração de energia elétrica em moendas de pequeno porte de até 1,2 cv.

Para que possa haver uma análise mais realista da participação do gás de síntese na conversão da energia deste sistema de gaseificação, tendo em vista que o objetivo é o incentivo ao uso da biomassa como fonte energética, devem ser feitos os ajustes na configuração do aparato experimental, visando uma diminuição na perda de carga do sistema.